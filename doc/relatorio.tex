\documentclass[a4paper, 11pt]{article}

\usepackage[portuges]{babel}
\usepackage[utf8]{inputenc}
\usepackage[margin=0.9in]{geometry}
\usepackage{amsmath}
\usepackage{graphicx}
\usepackage{datetime}
\usepackage{enumerate}
\usepackage{multicol}
\renewcommand{\baselinestretch}{1.0}

\emergencystretch 1pt%

\title{Relatório EA876 - Trabalho 1}
\author{Larissa Medeiros (178014) e Rafael Gonçalves (186062)}
\date{}
\begin{document}

\maketitle
\begin{multicols*}{2}

\section*{Introdução}

    O objetivo do trabalho foi fazer um compilador de expressões matemáticas para código assembly executável em processador de arquitetura ARM usando as linguagens Lex e Yacc.

A nossa abordagem consistiu em criar uma sintaxe livre de contexto que cobrisse todos os casos de expressões matemáticas a serem tratados e então resolver os casos base.

\section*{Método}

Selecionamos quais elementos precisariam ser identificados pelo Lex para a criação de tokens que seriam usados no Yacc. As regras no Yacc foram definidas segundo a seguinte gramática livre de contexto:

\begin{multicols*}{2}
\setlength{\columnsep}{0em}
\begin{enumerate}
\item
S $\rightarrow$ E

\item
E $\rightarrow$ N

\item
E $\rightarrow$ (E)

\item
E $\rightarrow$ -E

\item
E $\rightarrow$ E Op E

\item
Op $\rightarrow$ +

\item
Op $\rightarrow$ -

\item
Op $\rightarrow$ *

\item
N $\rightarrow$ [0-9]*

\end{enumerate}

\end{multicols*}

As principais expressões tratadas foram as que se resolvem para o símbolo E, já que isso significa efetivamente a resolução de expressões de forma recursiva.

A primeira expressão da gramática apenas define que o programa se resume a um símbolo intermediário de expressão, necessariamente. As rotinas de resolução dos casos 3, 4, 6, 7, 8 e 9 são simplesmente atribuições. Os demais casos imprimem instruções assembly, como mostradosa seguir:

\begin{multicols*}{2}
\bf Caso 2:
\begin{verbatim}
mov r0, #N
str r0, [SP], $4
\end{verbatim}

\break
\vfill

Caso 5:
\begin{verbatim}
ldr r2, [SP, #-4]!
ldr r1, [SP, #-4]!
OP r0, r1, r2
str r0, [SP], #4
\end{verbatim}

\end{multicols*}

Em que OP pode ser: add, mul ou sub, N é um número inteiro e SP é o stack pointer, no nosso caso o r10.

Tratamos o problema do armazenamento dos operandos na própria linguagem assembly. Utilizamos os comandos STR para armazenar o resultado de cada expressão intermediária na pilha e LDR para carregar resultados da pilha nos registradores utilizados nas operações.

\section*{Resultados}

Como proposto, nosso programa não resolve as expressões em C, mas sim imprime o código assembly que resolve a expressão.

Segue o código produzido pelo exemplo: $$(-5)*(5 + 2) - 7$$

\begin{multicols*}{2}

\begin{verbatim}
mov r10, #3200
mov r0, #5
str r0, [r10], #4
ldr r1, [r10, #-4]!
sub r0, #1, r1
str r0, [r10], #4
mov r0, #5
str r0, [r10], #4
mov r0, #2
str r0, [r10], #4
ldr r2, [r10, #-4]!
ldr r1, [r10, #-4]!
add r0, r1, r2
str r0, [r10], #4
mov r0, #7
str r0, [r10], #4
ldr r2, [r10, #-4]!
ldr r1, [r10, #-4]!
sub r0, r1, r2
str r0, [r10], #4
bl mul
str r0, [r10], #4
end

mul
ldr r2, [r10, #-4]!
ldr r1, [r10, #-4]!
mov r0, #0

loop
cmp r1, #0
ble endmul
add r0, r0, r2
sub r1, r1, #1
b loop

endmul
mov pc, lr

\end{verbatim}

\end{multicols*}

\end{multicols*}
\end{document}


